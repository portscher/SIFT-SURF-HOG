\UseRawInputEncoding
\documentclass{scrartcl}
\usepackage[utf8]{inputenc}
\usepackage{color}
\usepackage{csquotes}
\usepackage{graphicx}
\usepackage{enumitem}
\usepackage{lscape}
\usepackage{booktabs}
\usepackage{hyperref}
\usepackage{float}
\usepackage{xcolor}


\setcounter{secnumdepth}{0}

\title{Image classification with SVM}
\author{Andrea Portscher, Juan Pablo Stumpf, Philip Wille}

\begin{document}
\maketitle

\section{Introduction}
For this project we implemented an image classifier using three different feature extraction approaches: Scale-invariant feature transform (SIFT), Speeded up robust features (SURF) and Histogram of Gradients (HOG). First, we prepared an image dataset containing five different image classes, each comprising training and testing data. After extracting features using the above algorithms, we trained a Support-Vector Machine (SVM) and testetd, how well it was able to classify images.
\section{Approaches}
In the following section we will shortly explain the functionality of the feature extraction algorithms we applied to the dataset - namely SIFT, SURF and HOG.
In general, feature extraction algorithms work in three steps\cite{bay2006}:
\begin{enumerate}
  \item They select interest points at distinctive locations, such as corners.
  \item The neighbourhood of the interest points is represented by a feature vector.
  \item The descriptor vectors are matched between different images.
\end{enumerate}

\subsection{SIFT}
\subsection{SURF}
SURF is based on similar properties as SIFT but is - as the name already takes away - faster and more robust. Unlike SIFT, the SURF algorithm is currently still patented and can therefore only be used by including the \texttt{opencv\_contrib} package\footnote{https://github.com/opencv/opencv\_contrib} and allowing non-free algorithms.


\subsection{HOG}
\section{Implementation}
\subsection{Data}
The image data we used stems from the Caltech-256 Image Set\footnote{http://www.vision.caltech.edu/Image\_Datasets/Caltech256/, accessed on the 22nd of December 2020}, which consists of 256 sets of images of a certain class. We randomly selected 5 of these, namely: cactus, dice, raccoon, spaghetti and sushi. The preaparation of the image data was comprised of the following steps, which we implemented in the file \texttt{utils.py}:
\begin{itemize}
  \item All images were resized to the same size, converted to grayscale and normalized (which removes noise from the image).
  \item All images are associated with a label (their image class).
  \item The images are split into a training set (80\% of images per class) and a testing set (20\% of images).
\end{itemize}

\subsection{Pipeline}

\section{Discussion of Results}
Add confusion matrices


\bibliographystyle{unsrt}
\bibliography{sources}
\end{document}
